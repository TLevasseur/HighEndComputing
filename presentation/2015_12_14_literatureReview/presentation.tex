% Copyright 2004 by Till Tantau <tantau@users.sourceforge.net>.
%
% In principle, this file can be redistributed and/or modified under
% the terms of the GNU Public License, version 2.
%
% However, this file is supposed to be a template to be modified
% for your own needs. For this reason, if you use this file as a
% template and not specifically distribute it as part of a another
% package/program, I grant the extra permission to freely copy and
% modify this file as you see fit and even to delete this copyright
% notice. 

\documentclass{beamer}

% There are many different themes available for Beamer. A comprehensive
% list with examples is given here:
% http://deic.uab.es/~iblanes/beamer_gallery/index_by_theme.html
% You can uncomment the themes below if you would like to use a different
% one:
%\usetheme{AnnArbor}
%\usetheme{Antibes}
%\usetheme{Bergen}
%\usetheme{Berkeley}
%\usetheme{Berlin}
%\usetheme{Boadilla}
%\usetheme{boxes}
%\usetheme{CambridgeUS}
%\usetheme{Copenhagen}
%\usetheme{Darmstadt}
%\usetheme{default}
%\usetheme{Frankfurt}
%\usetheme{Goettingen}
%\usetheme{Hannover}
%\usetheme{Ilmenau}
%\usetheme{JuanLesPins}
%\usetheme{Luebeck}
\usetheme{Madrid}
%\usetheme{Malmoe}
%\usetheme{Marburg}
%\usetheme{Montpellier}
%\usetheme{PaloAlto}
%\usetheme{Pittsburgh}
%\usetheme{Rochester}
%\usetheme{Singapore}
%\usetheme{Szeged}
%\usetheme{Warsaw}

\title{Graph Partitionning}

% A subtitle is optional and this may be deleted
\subtitle{A review of three papers}

\author[T.~Levasseur]{Q.~Diaferia \and T.~Levasseur \and W.~Pei \and G.~Perez Bada}
% - Give the names in the same order as the appear in the paper.
% - Use the \inst{?} command only if the authors have different
%   affiliation.

\institute[Cranfield University] % (optional, but mostly needed)
{
	School of Aerospace\\
	Cranfield University
}
% - Use the \inst command only if there are several affiliations.
% - Keep it simple, no one is interested in your street address.

\date{High End Computing}
% - Either use conference name or its abbreviation.
% - Not really informative to the audience, more for people (including
%   yourself) who are reading the slides online

\subject{Theoretical Computer Science}
% This is only inserted into the PDF information catalog. Can be left
% out. 

% If you have a file called "university-logo-filename.xxx", where xxx
% is a graphic format that can be processed by latex or pdflatex,
% resp., then you can add a logo as follows:

% \pgfdeclareimage[height=0.5cm]{university-logo}{university-logo-filename}
% \logo{\pgfuseimage{university-logo}}

% Delete this, if you do not want the table of contents to pop up at
% the beginning of each subsection:
\AtBeginSection[]
{
	\begin{frame}<beamer>{Outline}
		\tableofcontents[currentsection,currentsubsection]
	\end{frame}
}

% Let's get started
\begin{document}
	
	\begin{frame}
		
		\titlepage
		
	\end{frame}
	
	\subsection<presentation>*{References}
	
	\begin{frame}[allowframebreaks]
		\frametitle<presentation>{References}
		
		\begin{thebibliography}{10}
			
			\beamertemplatearticlebibitems
			% Followed by interesting articles. Keep the list short. 
			
			\bibitem{Someone2000}
			G.~Karipys and V.~Kumar
			\newblock Multilevel graph partitioning schemes.
			
			\bibitem{Someone2000}
			G.~Karipys and V.~Kumar
			\newblock Multilevel k-way partitioning schemes for irregular graphs.
			
			\bibitem{Someone2000}
			G.~Karipys and V.~Kumar
			\newblock A fast and high quality multilevel schemes for partitioning irregular graphs.
			
		\end{thebibliography}
	\end{frame}
	
	
	
	% Section and subsections will appear in the presentation overview
	% and table of contents.
	\section{Coarsening}
	
	
	\begin{frame}{Coarsening phase}
		
		\begin{block}{Definition}
			Corasening aims to reduce the size of the graph in order to make it easier to split step by step with
			\(G_i \to G_{i+1}\)~and \(V^\nu_i \to \nu\) 
			\( with \begin{cases}
			weight(\nu)& = \Sigma ~ weight(V^\nu_i) \\
			edges(\nu) & =  \cup ~ edges(V^\nu_i)
			\end{cases}
			\)
		\end{block}
		\begin{block}{2 approaches}
			\begin{itemize}
				\item random matching and collapsing
				\item match and colapse vertice high connected node
			\end{itemize}
		\end{block}
	\end{frame}
	
	
	\begin{frame}{Random Matching (RM)}
		\begin{block}{Step by step}
			\begin{itemize}
				\item visit vertices in random order
				\item try to find a random unmatch neighbour
				\item if found, match them
				\item if not, only copy to the next graph step
			\end{itemize}
		\end{block}
	\end{frame}
	
	
	\begin{frame}{Heavy Edge Matching (HEM)}
		\begin{block}{Step by step}
			\begin{itemize}
				\item visit vertices in random order
				\item try to find an unmatch neighbour with the heavier edge
				\item if found, match them
				\item if not, only copy to the next graph step
			\end{itemize}
		\end{block}
	\end{frame}
	
	
	\begin{frame}{Modified Heavy Edge Matching (HEM*)}
		\begin{block}{Step by step}
			\begin{itemize}
				\item visit vertices in random order
				\item try to find an unmatch neighbour whose neigbours egdes towards the first vertice sum is maximal
				\item if found, match them
				\item if not, only copy to the next graph step
			\end{itemize}
		\end{block}
	\end{frame}
	
	
	\begin{frame}{Light Edge Matching (LEM)}
		\begin{block}{Step by step}
			\begin{itemize}
				\item visit vertices in random order
				\item try to find an unmatch neighbour with the lightest edge
				\item if found, match them
				\item if not, only copy to the next graph step
			\end{itemize}
		\end{block}
	\end{frame}
	
	
	\begin{frame}{Heavy Clique Matching (HCM)}
		\begin{block}{Step by step}
			\begin{itemize}
				\item find fully connected subgraph (or almost)
				\item match all the vertices of each subgraph
			\end{itemize}
		\end{block}
	\end{frame}
	\author[W.~Pei]{Q.~Diaferia \and T.~Levasseur \and W.~Pei \and G.~Perez Bada}
	\section{Initial partitioning}
	
	\begin{frame}{Spectral Bisection}
		
		\[a_{ij} =
		\begin{cases}
		ew(v_{i},v_{j})     & \quad \text{if } (v_{i},v_{j})  \in E_{m}\\
		0 & \quad \text{otherwise}\\
		\end{cases}
		\]
		\[ D: d_{ij} = \sum ew(v_{i},v_{j}) \]
		\[ Q = D-A \]
		Find eignevector y
		\[ P_{ij} =
		\begin{cases}
		1    & \quad \text{if } y_{i} < r \\
		2 & \quad \text{otherwise}\\
		\end{cases}
		\]
	\end{frame}
	
	
	
	\begin{frame}{KL algorithm}
		\begin{itemize}
			\item Bi partition
			\item Look for subset of vertexs in each partition such that swapping them leads to a partition with smaller edge-cut
		\end{itemize}
		\begin{block}{Gain}
			$gv(v) = \sum w(v,u)_{(v,u)\in E\bigwedge P[v]\neq P[u]} - \sum w(v,u)_{(v,u)\in E\bigwedge P[v]= P[u] }$
		\end{block}
	\end{frame}
	
	
	
	
	
	\begin{frame}{GGP}
		\begin{itemize}
			\item Pick a vertex
			\item Grow a region from it until half of the vertexes have been included
		\end{itemize}
	\end{frame}
	\begin{frame}{GGGP}
		\begin{itemize}
			\item Pick a vertex
			\item Includes vertexes that lead
			to the smaller increase in the edge-cut
		\end{itemize}
	\end{frame}
	
	
	\begin{frame}{Caraterization of different graph partitioning schemes}
		\begin{itemize}
			\item Number of trial
			\item Needs coordinates
			\item Quality
			\item Local view
			\item Global view
			\item Run time
			\item Degree of parallism
		\end{itemize}
	\end{frame}
	\author[G.~Perez Bada]{Q.~Diaferia \and T.~Levasseur \and W.~Pei \and G.~Perez Bada}
	\section{Uncoarsening}
	\begin{frame}{Uncoarsening}
		\begin{block}{Definition}
			Projects the partition \(P_{m}\) of the coarser graph to the original \\
			Goes step by step using \(G_{m-1}~G_{m-2}~...~G_1\) \\
			\(P_i~is~obtained~from~P_{i+1}\) by assigning the vertices collapsed from \(G_i\)\\
		\end{block}
		\begin{block}{Refinement algorithm}
			When obtaining \(P_i\), there are some degrees of freedom for distributing the unfolded nodes\\
			The way to distribute the nodes can boost the partitioning quality
		\end{block}
	\end{frame}
	
	\begin{frame}{Kernighan-Lin Refinement}
		\begin{block}{Base Principle}
			Being a graph split in partitions A and B, we select
			\[A'\subset A~and~B'\subset B~so~that~A\setminus A'\cup B~and~B\setminus B'\cup A\]\\
			Is a bisection with smaller edge cut count
		\end{block}
		\begin{block}{Key Concepts}
			\begin{itemize}
				\item Gain - Number of edge cuts decreased by swapping the vertex v
				\item Vertex move max (X) - Number of pointless iterations needed for the algorithm to stop
			\end{itemize}
		\end{block}
		\begin{block}{Algorithm Quality}
			\begin{itemize}
				\item Gain Computation - On which vertices will the gain be computed in each iteration
				\item Data Structure - For storing data relative to each vertex
			\end{itemize}
		\end{block}
	\end{frame}
	\begin{frame}{Refinement Methods}
		\begin{block}{Kernighan-Lin Refinement - KLR}
			Uses the partition obtained for \(G_{i+1}\) as the initial iteration partition for \(G_{i}\)\\
			Reaches convergence in less iterations (Usually around five)\\
			Low number of vertex swaps
		\end{block}
		\begin{block}{Greedy Refinement - GR}
			Modification of KLR taking into account that most of the progress is done in the first iteration\\
			Performs a single iteration for each partition step\\
			Reduced computation time
		\end{block}
	\end{frame}
	\begin{frame}{Refinement Methods}	
		\begin{block}{Boundary Refinement}
			Only take into account boundary vertices\\
			If a vertex is swapped, recalculate adjacent gains and update them to boundary if needed		
			\begin{itemize}
				\item \textbf{Greedy - BGR} - Performs a single iteration
				\item \textbf{Kernighan-Lin - BKLR} - Iteration number defined by accuracy
				\item \textbf{Hybrid - BKLGR} - Uses BKLR for small graphs and BGR for bigger ones
			\end{itemize}
		\end{block}
	\end{frame}
	\begin{frame}{K-way Refinement Methods}
		The complexity of the KL algorithm increases when using more than 2 partitions \\
		\textbf{Neighborhood N(v)} - The external partitions connected to each vertex\\
		\textbf{Load Balancing condition} - Minimum and maximum work loads are assigned for every partition\\
		\textbf{Priority lists} - \(k*(k-1)\) 
		\begin{block}{K-way Greedy Refinement - kGR}
			Eliminates lookahead - Vertex cluster swapping\\
			No need for priority lists
		\end{block}
		\begin{block}{K-way Global LR Refinement - kGKLR}
			Fixes the local minima nature of the algorithm with a global priority queue\\
			Each vertex with at least one non-negative gain gets added to the queue and it is swapped to the partition with
			the highest gain that addresses the Load Balancing Condition 
		\end{block}
	\end{frame}
	\author[Q.~Diaferia]{Q.~Diaferia \and T.~Levasseur \and W.~Pei \and G.~Perez Bada}
	\section{Results}
	
	\begin{frame}{Coarsening phase}
		\begin{block}{First article}
			HEM :
			\begin{itemize}
				\item Good initial partition, smaller edge-cut, little refinement
				\item Requires more time
			\end{itemize}
		\end{block}
		\begin{block}{Second article}
			HEM :
			\begin{itemize}
				\item Good initial partition, smallest overall runtime
			\end{itemize}
			HEM* :
			\begin{itemize}
				\item Better for finite element meshes
			\end{itemize}
		\end{block}
		\begin{block}{Third article}
			HEM :
			\begin{itemize}
				\item Good initial partition, smallest overall runtime
			\end{itemize}
		\end{block}
	\end{frame}
	
	\begin{frame}{Initial partitioning phase}
		\begin{block}{First and third articles}
			GGGP :
			\begin{itemize}
				\item Smaller edge-cut and runtime
			\end{itemize}
		\end{block}
	\end{frame}
	
	\begin{frame}{Uncoarsening phase}
		\begin{block}{First article}
			BKLGR :
			\begin{itemize}
				\item Good balance between small edge-cut and fast execution
			\end{itemize}
		\end{block}
		\begin{block}{Second article}
			kGR and kGKLR :
			\begin{itemize}
				\item similar edge-cut
				\item kGKLR requires up to 2 times more time
			\end{itemize}
		\end{block}
		\begin{block}{Third article}
			BKLGR :
			\begin{itemize}
				\item Good balance between small edge-cut and fast execution
			\end{itemize}
		\end{block}
	\end{frame}
	
	
	% All of the following is optional and typically not needed. 
	\appendix
	\section<presentation>*{\appendixname}
	\subsection<presentation>*{References}
	
	\begin{frame}[allowframebreaks]
		\frametitle<presentation>{References}
		
		\begin{thebibliography}{10}
			
			\beamertemplatearticlebibitems
			% Followed by interesting articles. Keep the list short. 
			
			\bibitem{Someone2000}
			G.~Karipys and V.~Kumar
			\newblock Multilevel graph partitioning schemes.
			
			\bibitem{Someone2000}
			G.~Karipys and V.~Kumar
			\newblock Multilevel k-way partitioning schemes for irregular graphs.
			
			\bibitem{Someone2000}
			G.~Karipys and V.~Kumar
			\newblock A fast and high quality multilevel schemes for partitioning irregular graphs.
			
		\end{thebibliography}
	\end{frame}
	
\end{document}
